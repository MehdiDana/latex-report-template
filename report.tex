% Template for report using LaTex
\documentclass[11pt,a4paper]{report}
\usepackage{graphicx}
\usepackage{wrapfig}
\usepackage{fullpage}
\usepackage{fancyhdr}
\usepackage{multicol}
\usepackage{listings}
\usepackage{color}
\usepackage{xcolor}
\usepackage{caption}
\usepackage{verbatim}
\usepackage{mathtools}

\definecolor{dkgreen}{rgb}{0,0.6,0}
\definecolor{gray}{rgb}{0.5,0.5,0.5}
\definecolor{mauve}{rgb}{0.58,0,0.82}
\lstset{frame=tb,
  language=Prolog,
  aboveskip=3mm,
  belowskip=3mm,
  showstringspaces=false,
  columns=flexible,
  basicstyle={\small\ttfamily},
  numbers=none,
  numberstyle=\tiny\color{gray},
  keywordstyle=\color{blue},
  commentstyle=\color{dkgreen},
  stringstyle=\color{mauve},
  breaklines=true,
  breakatwhitespace=false,
  tabsize=3 
}

\setlength{\headheight}{15.2pt}
\headsep = 25pt
\pagestyle{fancy}
\lhead{Left heading} % To be changed
\rhead {Right heading} % To be changed
\lfoot{University of Huddersfield } % To be changed
\rfoot{Right footer } % To be changed
\usepackage[pdfauthor={Mehdi Dana},% Change to your name
pdfkeywords={LaTex report template - University of Huddersfield},%
pdftitle={LaTex report template },% Change to your document title
pdfsubject={Report },% % Change to your subject area
pagebackref=true,%
pdftex]{hyperref}
\usepackage{hyperref}
\hypersetup{
    colorlinks,%
    citecolor=black,% change citation reference colour
    filecolor=black,%
    linkcolor=black,% change links colour
    urlcolor=blue 
}

\begin{document}
% This is the first page, please change it to your information
\title{ A report template in \LaTeX{} for students }
\author{Your name \\ 
Email: username@hud.ac.uk \\
Supervisor:  \{Your supervisor name;  their emails \} \\
School of Computing and Engineering, University of Huddersfield \\
%\texttt{ Module name}  
%\thanks {Thanks to: Your module lecturer  } \\
}
\date{ date } % date 
\maketitle

\begin{abstract}
This is a report template in \LaTeX{} for students in University of Huddersfield, see university logo in figure \ref{fig: hud_fig} (example of referencing to figure). \\
You will find some other example such as \textit{including figure} and \textit{listing}.

\end{abstract}

\chapter*{Acknowledgement}
% This is an example of acknowledgement
I am grateful to my supervisor, Prof. X for his advice and guidance. \\

\tableofcontents % show table of contents

\listoffigures	% show list of figures

\chapter{Introduction}
\label{Introduction}
\newpage
\section{Summary}
The purpose of this paper was to help students use \LaTeX{} because of many reasons:
\begin{itemize}
\item Latex has user friendly interface
\item In latex, you get what you mean
\end{itemize}
This is introduction. Lets start with some quotes \\
Joy in looking and comprehending is nature's most beautiful gift \footnote{Albert Einstein}.

\newpage
\section{Motivation}
I am inspired by this quote from Confiuos; "Find a job you love, you'll never work a day in your life."


\chapter{Methodology}
\label{Methodology}
\newpage
\section{Assumption}
We assumed that everyone who uses this template has at least basic knowledge of latex.
\bigskip
\begin{figure}[hb]
\centering
\includegraphics[width=6in]{figures/dream.png} 
\caption{Example of a figure }
% explaination 
This figure illustrates ...  
\label{fig: dream}
\end{figure}

\section{Mathematics}
It is easy to show a matrix in latex, for example we write 
$\begin{bmatrix}
	0 & 1 & 2 \\
	1 & 2 & 1 \\
	0 & 2 & 3
\end{bmatrix}$
in the same line, however we could write 
\[
	A = \begin{bmatrix}
		0 & 1 & 2 \\
		1 & 2 & 1 \\
		0 & 2 & 3  
	\end{bmatrix}
\]
Here is citation \cite{Prof_Lee}

\newpage
\section{Programming}
\subsection{Java}
In \LaTeX{} you can include Java code, for example 
\lstset{numbers=left, stepnumber=2, caption= HelloWorld.Java, label=HelloWorld}
\lstinputlisting{codes/HelloWorld.java}
Similarly, it is easy to include any other programming languages such as C++ and PHP.
\\ \\
You could write java code directly into file, for example listing \ref{HelloWorld2} is not included.  
\lstset{numbers=left, stepnumber=2, caption= HelloWorld2, label=HelloWorld2}
\begin{lstlisting}
	/* try and catch example */
	try{
		System.out.println("Hello World");
	}catch (Exception e){
		System.out.println("Failed to print");
	}
\end{lstlisting}

\subsection{Logic}
It is \emph{hard} to write this in word documents: 
\[
 \forall x \in X, \quad \exists y \leq \epsilon
\]


\chapter{Conclusion}
\section{Summary}
This was a report template in latex, we tried to keep it simple so you could understand and change it to meet your requirement.

\chapter{Reference and Appendix}
\newpage

% please run bibtex in order to see references in your documents
\addcontentsline{toc}{section}{Bibliography}
\nocite{*}
\bibliographystyle{plain}		% (uses file "plain.bst")
\bibliography{references}		% expects file "myrefs.bib"

\newpage
\addcontentsline{toc}{section}{Appendices}
\section*{Appendices}
\bigskip
\begin{figure}[hb]
\centering
\includegraphics[width=6in]{figures/hud.png} 
\caption{ University of Huddersfield logo  }
% explaination 
This is logo of Huddersfield University  
\label{fig: hud_fig}
\end{figure}

\end{document}
% Copyright: University of Huddersfield [Prof. Lee McCluskey t.l.mccluskey@hud.ac.uk  and Mehdi Dana u0960731@unimail.hud.ac.uk]